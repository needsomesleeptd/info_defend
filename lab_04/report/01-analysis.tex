\chapter{Аналитический раздел}

\section{Электронная цифровая подпись}

Электронная (цифровая) подпись позволяет подвердить авторство электронного документа. Она связана не только с автором документа, но и с самим документов (при помощи криптографических методов) и не может быть подделана при поммощи обычного копирования.

Создание ЭП с использованием криптографического алгоритма RSA и алгоритма хеширования SH1/MD5 происходит следующим образом:
\begin{enumerate}
	\item происходит хеширование сообщения при помощи SH1/MD5, сообщение~--- файл, который неообходимо подписать;
	\item происходит шифрование с использованием закрытого ключа RSA последовательности 128/160 бит, полученных на предыдущем этапе;
	\item значение подписи~--- результат шифрования.
\end{enumerate}

Проверка ЭП с использованием криптографического алгоритма RSA и алгоритма хеширования SH1 происходит следующим образом:
\begin{enumerate}
	\item происходит хеширование сообщения при помощи SH1/MD5, сообщение~--- файл, подпись которого необходимо проверить;
	\item происходит расшифровка
	подписи с использованием открытого ключа RSA;
	\item происходит побитовая сверка значений, полученных на предыдущих этапах, если они одинаковы, подпись считается подлинной.
\end{enumerate}


\section{Алгоритм RSA}

RSA~--- криптографический алгоритм с открытым ключом, разработанный учеными Ривестом, Шамиром и Адлеманом, основывающийся на вычислительной сложности задачи факторизации больших целых чисел.

Криптографическая система с открытым ключом~--- система шифрования, при которой открытый ключ передаётся по незащищенному каналу и используется для проверки электронных подписей и для шифрования сообщения.
Для генерации электронной подписи и расшифровки сообщений используется закрытый ключ.

Алгоритм RSA состоит из следующих этапов~\cite{rsa}:
\begin{enumerate}
	\item Создание открытого и закрытого ключа:
	\begin{itemize}
		\item выбирается два простых числа $p$ и $q$;
		\item вычисляется $n = p \cdot q$;
		\item вычисляется функция Эйлера $\varphi(n) = (p - 1) \cdot (q - 1)$;
		\item выбирается открытая экспонента $e \in (1; \varphi(n))$;
		\item вычисляется закрытая экспонента $d$, где $(d \cdot e) \text{ mod } \varphi(n) = 1$;
		\item создается открытый ключ $(e, n)$ и закрытый ключ $(d, n)$;
	\end{itemize}
	\item Шифрование и дешифрование:
	\begin{itemize}
		\item шифрование сообщения: $w = G(h) = h^e \text{ mod } n$;
		\item дешифрование сообщения: $h = Q(w) = w^d \text{ mod } n$;
	\end{itemize}
\end{enumerate}

Надежность шифрования обеспечивается тем, что третьему лицу (старающемуся взломать шифр) очень трудно вычислить закрытый ключ по от крытому.
Оба ключа вычисляются из одной пары простых чисел (v и u).
Если число является произведением двух очень больших простых чисел, что его трудно разложить на множители~\cite{rsa}.

\section{Алгоритм SHA-1}

SHA-1 (англ. Secure Hash Algorithm 1) — это алгоритм хеширования, предназначенный для получения последовательности длиной 160 бит, используемой для проверки целостности и подлинности сообщений произвольной длины.

На вход алгоритма поступает последовательность бит произвольной длины $ L $, хеш которой нужно найти.

Алгоритм SHA-1 состоит из 4 следующих этапов:

\begin{enumerate}
	\item выравнивание потока;
	\item добавление длины сообщения;
	\item инициализация буфера;
	\item вычисления в цикле.
\end{enumerate}

Выравнивание потока заключается в добавлении некоторого числа нулевых бит, чтобы новая длина последовательности $ L' $ стала сравнима с 448 по модулю 512. 

После этого добавляется 64-битное представление длины исходного сообщения (в битах) в конец выровненной последовательности.

Сначала записывают младшие 8 байтов, затем старшие.

Далее происходит инициализация буфера, состоящего из 5 переменных $ A, B, C, D, E $ размерностью 32 бита, начальные значения которых задаются шестнадцатеричными числами (порядок от младших к старшим).

В этих переменных будут храниться результаты промежуточных вычислений.

Далее в цикле каждый блок длиной 512 бит проходит 80 раундов вычислений. Для этого блок представляется в виде массива $ W $ из 80 слов по 32 бита. Первые 16 слов копируются из блока, а оставшиеся 64 слова формируются по специфическому правилу, основанному на предыдущих значениях.

Все раунды имеют однотипную структуру и могут быть описаны следующим образом:

$$ A = B + \left( \text{LeftRotate}(A, 5) + \text{F}(B, C, D) + W[t] + K \right) $$

где $ t $ — текущий номер раунда (от 0 до 79), 

$ \text{LeftRotate} $ — операция циклического сдвига влево, 

$ F $ — нелинейная функция, определяющая раунд, 

$ W[t] $ — слово из массива, 

$ K $ — константа, специфичная для каждого раунда.

Результат вычислений хранится в переменных $ A, B, C, D, E $.
