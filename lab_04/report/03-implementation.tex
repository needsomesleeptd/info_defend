\chapter{Технологический раздел}

\section{Средства реализации}

В качестве языка программирования, используемого при написании данной лабораторной работы, был выбран C++~\cite{cpp-lang}.

\section{Реализация алгоритмов}

В листингах~\ref{lst:RSA.h}~--~\ref{lst:MD5.cpp} представлены реализации разрабатываемых модулей.
\includelistingpretty
	{RSA.h}
	{c++}
	{Реализация класса алгоритма RSA (получение ключей)}
	
\includelistingpretty
	{RSA.cpp}
	{c++}
	{Реализация класса алгоритма RSA}

\includelistingpretty
	{MD5.cpp}
	{c++}
	{Реализация алгоритма MD5}

\clearpage

\section{Тестирование}
В таблице~\ref{tbl:functional_test} представлены функциональные тесты.
\begin{table}[ht!]
	\begin{center}
		\captionsetup{justification=raggedright,singlelinecheck=off}
		\caption{\label{tbl:functional_test} Функциональные тесты}
		\begin{tabular}{|m{1em}|m{15em}|m{15em}|}
			\hline
			№ & Входные данные & Выходные данные \\ 
			\hline
			1 & 2 пустых файла & Signature is valid \\
			\hline
			2 & 2 одинаковых текстовых файла & Signature is valid \\
			\hline
			3 & 2 разных текстовых файла & Signature is invalid \\
			\hline
			4 & 2 одинаковых архива & Signature is valid \\
			\hline
			5 & 2 разных архива & Signature is invalid \\
			\hline
		\end{tabular}
	\end{center}
\end{table}